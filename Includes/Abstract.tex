% Resumo em l�ngua estrangeira (em ingl�s Abstract, em espanhol Resumen, em franc�s R�sum�)
\begin{center}
	{\Large{\textbf{\ForeignTitle}}}
\end{center}

\vspace{1cm}

\begin{flushright}
	Author: \Author\\
	Advisor: \Advisor\\
	Co-advisor: \Coadvisor
\end{flushright}

\vspace{1cm}

\begin{center}
	\Large{\textsc{\textbf{Abstract}}}
\end{center}

\noindent Real-time strategy games (RTS) have become target of many scientific works over time due to their complexity and, consequently, to the great difficulty in creating effective strategies using artificial intelligence techniques. Among these games, StarCraft has excelled as a great reasearch laboratory. This work presents a model to manage resources in these applications. Once RTS games are an abstraction of the war environments, the proposed model is based on a organizational model to determine the areas of operation of each character in the game. This model was applied in StarCraft and experimented with some existing scenarios of this game. The obtained results were compaired to results from methods used by professional players. These results show that the strategies used by the proposed model are consistent with the techniques used by professional players.

\noindent\textit{Keywords}: Real-time Strategy Games, Non-Playable Characters, Resource Management.