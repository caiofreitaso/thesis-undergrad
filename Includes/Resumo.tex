% Resumo em l�ngua vern�cula
\begin{center}
	{\Large{\textbf{\Title}}}
\end{center}

\vspace{1cm}

\begin{flushright}
	Autor: \Author\\
	Orientador: \Advisor\\
	Co-orientador: \Coadvisor
\end{flushright}

\vspace{1cm}

\begin{center}
	\Large{\textsc{\textbf{Resumo}}}
\end{center}

\noindent Jogos de estrat�gia em tempo real (RTS) se tornaram foco de diversos trabalhos cient�ficos ao longo do tempo devido � grande complexidade envolvida neles e, consequentemente, uma grande dificuldade em criar estrat�gias eficazes com o uso de t�cnicas de intelig�ncia artificial. Dentre estes jogos, StarCraft tem se destacado como um �timo laborat�rio de pesquisa. Este trabalho apresenta um modelo cujo objetivo � o gerenciamento de recursos nestas aplica��es. Como jogos RTS s�o uma abstra��o de um ambiente de guerra, o modelo proposto se baseia em um modelo organizacional para determinar as �reas de atua��o dos personagens no jogo. Este modelo foi implementado usando StarCraft como estudo de caso e experimentado com alguns cen�rios existentes deste jogo. Os resultados obtidos foram comparados com resultados obtidos de m�todos utilizados pelos jogadores profissionais. Estes resultados demonstram que as estrat�gias usadas pelo modelo proposto s�o coerentes com as t�cnicas usadas pelos jogadores profissionais.

\noindent\textit{Palavras-chave}: Jogos de Estrat�gia em Tempo Real, Personagens Inteligentes, Gerenciamento de Recursos.